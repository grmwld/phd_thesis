 % -*- root: ../main.tex -*-
\documentclass[../main.tex]{subfiles}
\begin{document}

\begin{chapter}{Conclusions et Perspectives}


\begin{section}{Conclusions}

L'ensemble de ce 

En conclusion, ma thèse à consisté à mettre en évidence les intéractions croisées entre les \glspl{ht} et les \glspl{gc} dans deux tissus au devenir différent.

Grande tissu spé de l'implémentation de fonctions bio affectées par les crosstalk, quels qu'ils soient.

Inclue des processus inattendus comme le système immun.

\end{section}


\begin{section}{Perspectives}

L'utilisation du \gls{rnaseq} s'est révélée dans le cadre de ce projet, particulièrement enrichissante pour aborder l'analyse de programmes de régulation croisées entre les \glspl{ht} et les \glspl{gc}.
Toutefois, il est difficile de tirer des conclusions définitives sur la biologie fine qui a lieu.
En particulier, la validation par d'autres technologies (\gls{rtqpcr}) de gènes qui semblent centraux dans la résultante des interactions \gls{ht} - \gls{gc} sont à réaliser.
\par
En outre, bien que ce travail mette en valeur les modalités d'interactions entre ces deux hormones et leurs effets sur le transcriptome, tout reste à faire en ce qui concerne l'élucidation des détails mécanistiques qui sous-tendent ces interactions croisées.
La tâche est d'autant plus difficile qu'à la vue de nos résultats, il semblerait qu'une grande diversité de mécanismes soient impliqués, de la compétition pour des co-régulateurs à une action synergique en \textit{cis} en passant par des voies non-génomiques.
\par
Parmi la multitude résultats obtenus, il apparaît une composante forte associée à une modulation de la méthylation de l'\gls{dna}.
Il reste à ce sujet à déterminer dans quelle mesures ces variations des niveaux de méthylation vont affecter certains gènes en particulier ou le génome dans son ensemble, et quelles en sont les conséquences fonctionnelles.



\end{section}

\end{chapter}

\end{document}