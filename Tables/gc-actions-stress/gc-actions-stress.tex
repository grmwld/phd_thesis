\setlength{\extrarowheight}{5px}

\begin{table}[!htbp]
\footnotesize

\def\tabularxcolumn#1{m{#1}}
\newcolumntype{L}{>{\raggedright\arraybackslash}X}
\newcolumntype{M}{>{\setlength\hsize{1\hsize}\raggedright}X}
\newcolumntype{N}{>{\setlength\hsize{1\hsize}\centering}X@{}}
\newcolumntype{O}{>{\setlength\hsize{1\hsize}\centering}X}
\newcolumntype{P}{>{\centering\setlength\hsize{2\hsize}}X}

\begin{tabularx}{\textwidth}{M N O N O}

\toprule

\textbf{Action}		& \multicolumn{2}{P}{\glspl{gc} médiateurs de la réponse au stress}
					& \multicolumn{2}{P}{\glspl{gc} temporisateurs de la réponse au stress} \tabularnewline

					\cmidrule(rl){2-3}				\cmidrule(rl){4-5}

\textbf{Effets}		& Permissive	& Stimulatrice	& Préparatrice	& Suppréssive \tabularnewline

Cardiovasculaires
					& \checkmark	& 				& 				& 			 \tabularnewline
Volumes de fluides
					& 				& 				& 				& \checkmark\textsuperscript{*} \tabularnewline
Système immun
					& \checkmark	& 				& 				& \checkmark\textsuperscript{*} \tabularnewline
Métabolisme
					& \checkmark	& \checkmark	& \checkmark	& 			 \tabularnewline
Transport du glucose et utilisation par le cerveau
					& 				& 				& 				& \checkmark\hphantom{\textsuperscript{*}} \tabularnewline
Appétit
					& 				& 				& \checkmark	& \checkmark\textsuperscript{*} \tabularnewline
Cognitifs
					& \checkmark	& 				& 				& \checkmark\hphantom{\textsuperscript{*}} \tabularnewline
Comportement et physiologie de la reproduction
					& \checkmark	& \checkmark	& \checkmark	& 			 \tabularnewline

\bottomrule

\end{tabularx}
\caption[Effets des glucocorticoïdes dans la réponse au stress]
{
Types d'effets physiologiques des \glspl{gc} dans la réponse au stress.
Tiré de \citet{Sapolsky2000}.
\checkmark\textsuperscript{*} désigne un effet supprésseur aussi bien par les niveaux de \glspl{gc} basaux que induits par le stress.
}
\label{tab:gc-actions-stress}

\def\tabularxcolumn#1{p{#1}}
\end{table}