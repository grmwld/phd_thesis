% BOTTOM caption
% ------------------------
\begin{figure}[!htb]
\centering
\vspace{1\baselineskip}
\includegraphics[width=\textwidth]
% ------------------------
%
% SIDE caption
% ------------------------
%\begin{SCfigure}[\sidecaptionrelwidth][!htbp]
%\centering
%\vspace{1\baselineskip}
%\includegraphics[width=0.5\textwidth]
% ------------------------
%
% Main information
% ===========================================================
{Figures/tr-isoforms/tr-isoforms.pdf}
\caption[Isoformes du récepteur aux hormones thyroïdiennes]
{
Les \glspl{tr} sont codés par deux gènes, \gls{tra} et \gls{trb}.
\gls{tra} peut générer quatre isoformes:
\gls{tra}-1 est la seule forme possédant un \gls{lbd} fonctionnel.
\gls{trda}-1 et -2 ainsi que \gls{tra}-2 jouent le rôle d'antagonistes ou de dominants négatifs.
Les deux isoformes de \glspl{trb} sont fonctionnelles mais ont des profils d'expression tissus spécifiques.
Seules les formes identifiées chez l'humain sont représentées ici.
NTD: \acrlong{ntd}; DBD: \acrlong{dbd}; LBD: \acrlong{lbd}.
Figure inspirée de \citet{Jones2003,Flamant2003,Gonzalez-Sancho2003,Bassett2009,Tagami2010}.
}
\label{fig:tr-isoforms}
% ===========================================================
%
% BOTTOM caption
% ------------------------
\end{figure}
% ------------------------
%
% SIDE caption
% ------------------------
%\end{SCfigure}
% ------------------------
%
%
%\missingfigure{Make a figure}