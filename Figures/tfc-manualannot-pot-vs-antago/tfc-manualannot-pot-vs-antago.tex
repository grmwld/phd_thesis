% BOTTOM caption
% ------------------------
\begin{figure}[!htbp]
\centering
\vspace{1\baselineskip}
\includegraphics[width=0.9\textwidth]
% ------------------------
%
% SIDE caption
% ------------------------
%\begin{SCfigure}[\sidecaptionrelwidth][!htbp]
%\centering
%\vspace{1\baselineskip}
%\includegraphics[width=0.5\textwidth]
% ------------------------
%
% Main information
% ===========================================================
{Figures/tfc-manualannot-pot-vs-antago/tfc-manualannot-pot-vs-antago.png}
\caption[Termes enrichis dans les catégories de "potentiation" et d'"antagonisme" dans l'épiderme caudal]
{
Termes enrichis dans les catégories de gènes "antagonisés" (à gauches) et "potentiés" (à droite) dans l'épiderme caudal.
La valeur absolue de l'axe des abscisses représente le nombre de gène associé à chaque terme (axe des ordonnées).
Seuls les 50 termes les plus représenté sont illustrés ici.
Les barres verticales rouge correspondent au nombre théorique de gènes associés à chaque terme dans le cas d'une répartition aléatoire entre profils "antagonisés" et "potentiés".
}
\label{fig:tfc-manualannot-pot-vs-antago}
% ===========================================================
%
% BOTTOM caption
% ------------------------
\end{figure}
% ------------------------
%
% SIDE caption
% ------------------------
%\end{SCfigure}
% ------------------------