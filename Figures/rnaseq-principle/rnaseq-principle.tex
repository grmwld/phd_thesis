% BOTTOM caption
% ------------------------
\begin{figure}[!htb]
\centering
\vspace{1\baselineskip}
\includegraphics[width=\textwidth]
% ------------------------
%
% SIDE caption
% ------------------------
%\begin{SCfigure}[\sidecaptionrelwidth][!htbp]
%\centering
%\vspace{1\baselineskip}
%\includegraphics[width=0.5\textwidth]
% ------------------------
%
% Main information
% ===========================================================
{Figures/rnaseq-principle/rnaseq-principle.pdf}
\caption[Principe du RNA-Seq]
{
Principe du RNA-Seq.
1) Les ARN totaux sont extraits des tissus (ici queue et bourgeon de membres postérieurs) et sélectionnés sur colonne poly-T.
2) Des \glspl{cdna} sont synthétisés à partir des \glspl{rna}, 3) et fragmentés.
Après séquençage 4) (dans le cadre de ce projet en chimie TruSeq), les "reads" sont repositionnés sur le génome, et comptés dans les modèles de gènes connus 5) afin d'en inférer l'expression différentielle.
}
\label{fig:rnaseq-principle}
% ===========================================================
%
% BOTTOM caption
% ------------------------
\end{figure}
% ------------------------
%
% SIDE caption
% ------------------------
%\end{SCfigure}
% ------------------------
%
%
%\missingfigure{Make a figure}