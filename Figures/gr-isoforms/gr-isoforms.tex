% BOTTOM caption
% ------------------------
\begin{figure}[!htb]
\centering
\vspace{1\baselineskip}
\includegraphics[width=\textwidth]
% ------------------------
%
% SIDE caption
% ------------------------
%\begin{SCfigure}[\sidecaptionrelwidth][!htbp]
%\centering
%\vspace{1\baselineskip}
%\includegraphics[width=0.5\textwidth]
% ------------------------
%
% Main information
% ===========================================================
{Figures/gr-isoforms/gr-isoforms.pdf}
\caption[Isoformes du récepteur aux glucocorticoïdes]
{
Il n'existe qu'un seul gène codant pour \gls{gr}.
Deux isoformes \gls{gra} et \gls{grb} sont obtenues par épissage alternatif du 9ème exon.
Contrairement à \gls{gra}, \gls{grb} ne possède pas de \gls{lbd} fonctionnel, et agit en tant que dominant négatif naturel.
}
\label{fig:gr-isoforms}
% ===========================================================
%
% BOTTOM caption
% ------------------------
\end{figure}
% ------------------------
%
% SIDE caption
% ------------------------
%\end{SCfigure}
% ------------------------
%
%
%\missingfigure{Make a figure}