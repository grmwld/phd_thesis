% BOTTOM caption
% ------------------------
\begin{figure}[!htb]
\centering
\vspace{1\baselineskip}
\includegraphics[width=\textwidth]
% ------------------------
%
% SIDE caption
% ------------------------
%\begin{SCfigure}[\sidecaptionrelwidth][!htbp]
%\centering
%\vspace{1\baselineskip}
%\includegraphics[width=0.5\textwidth]
% ------------------------
%
% Main information
% ===========================================================
{Figures/gc-modes/gc-modes.pdf}
\caption[Modes d'action généraux des glucocorticoïdes]
{\glsreset{gre}
Modes d'action généraux des \glspl{gc}.
Les \glspl{gc} sont produits par la glande corticosurrénale, libérés dans la circulation sanguine et diffusent dans les cellules des tissus cibles.
En absence de ligand, \gls{gr} est séquestré dans le cytoplasme par des protéines chaperonnes dont \gls{hsp90} et \gls{hsp70}.
La liaison du ligand sur \gls{gr} provoque la dissociation des complexes de séquestration et la translocation de \gls{gr} activé dans le noyau.
Il va pouvoir alors se fixer à l'ADN sur des motifs spécifiques correspondant à des éléments de réponse, les \glspl{gre}.
Les complexes co-régulateurs recrutés vont pouvoir servir de médiateurs de l'activité trans-répressive ou trans-activatrice de \gls{gr}.
}
\label{fig:gc-modes}
% ===========================================================
%
% BOTTOM caption
% ------------------------
\end{figure}
% ------------------------
%
% SIDE caption
% ------------------------
%\end{SCfigure}
% ------------------------
%
%
%\missingfigure{Make a figure}